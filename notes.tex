\documentclass[10pt]{article}

\usepackage{graphicx}
\usepackage{amssymb}
\usepackage[margin=1in,footskip=0.25in]{geometry}
\usepackage{comment}

\usepackage{hyperref}

\renewcommand{\ttdefault}{cmtt}

\begin{document}

\section{Probabilistic}

\subsection{Conditional probability and Bayesian formula}

Joint probability, written as $P(A\cap B)$ or $P(A,B)$, means two events happening at the
same time.  Conditional probability is defined as:
$$
P(A|B)\triangleq\frac{P(A\cap B)}{P(B)}
$$
Then
$$
P(A\cap B)=P(A|B)P(B)=P(B|A)P(A)
$$
thus Bayesian formula:
$$
P(A|B)=\frac{P(B|A)P(A)}{P(B)}
$$

\begin{comment}
An example. Let
$$
X=\left\{\begin{array}{ll}
1 & \mbox{if sunny} \\
0 & \mbox{otherwise}
\end{array}\right.
$$
$$
Y=\left\{\begin{array}{ll}
1 & \mbox{if warm} \\
0 & \mbox{if cold}
\end{array}\right.
$$
$$
\Pr\{\mbox{sunny}\}=\Pr\{\mbox{sunny and warm}\}+\Pr\{\mbox{sunny and cold}\}
$$
$$
\Pr\{X=x\}=\sum_y\Pr\{X=x,Y=y\}
$$
\end{comment}

\subsection{Likelihood and maximum likelihood estimation (MLE)}

A likelihood function or simply \emph{likelihood} is the probability of data
give parameters. It is a function of parameters.
$$
\mathcal{L}(\theta)=\Pr\{d|\theta\}
$$
MLE finds $\hat{\theta}$ that maximizes $\mathcal{L}(\theta)$.  The estimate
$\hat{\theta}$ satisfies:
$$
\left.\frac{\partial \mathcal{L}(\theta)}{\partial\theta}\right|_{\theta=\hat{\theta}}=0
$$
Or sometimes for the ease of derivaton:
$$
\left.\frac{\partial \log\mathcal{L}(\theta)}{\partial\theta}\right|_{\theta=\hat{\theta}}=0
$$

\begin{comment}
An example. Suppose in the last August, 20 out of 31 days were sunny.
\end{comment}

\subsection{The estimation-maximization algorithm (EM)}

For a complex problem, it is often difficult to write down the close form of
$\mathcal{L}(\theta)$, so we can't directly do derivation. EM is an iterative
algorithm to find MLE when the likihood function can be expressed with
latent/hidden variables.
$$
\mathcal{L}(\theta)=\Pr\{d|\theta\}=\sum_z\Pr\{d,Z=z|\theta\}
$$
where $\sum_{\bf z}$ sums over all latent states. For contiguous latent
variables, replace sum with integral. The central element in EM is the
$Q$-function, which is defined by:
$$
Q(\theta|\theta^t)=\mathbb{E}_{{\bf z}|{\bf d},\theta^t}\log\Pr\{{\bf d},{\bf Z}={\bf z}|\theta\}=\sum_{{\bf z}}\Pr\{{\bf Z}={\bf z}|{\bf d},\theta^t\}\cdot\log\Pr\{{\bf d},{\bf Z}={\bf z}|\theta\}
$$
where $\theta^t$ is the estimate at $t$-th iteration. For a standard EM, the
estimate at the $(t+1)$ iteration $\theta^{t+1}$ maximizes $Q(\theta|\theta^t)$:
$$
\left.\frac{\partial Q(\theta|\theta^t)}{\partial\theta}\right|_{\theta=\theta^{t+1}}=0
$$
It can be proved that $\theta^{t+1}$ increases $\mathcal{L}(\theta)$.

For a problem solvable by EM, $Q(\theta|\theta^t)$ can usually be expressed in
the close form or take a simpler form than $\mathcal{L}(\theta)$. EM guarantees
convergence, but doesn't guarantee to converge to the global maxmium.

\section{Resolving ambiguously mapped reads for RNA-seq}

Suppose we have $N$ transcripts and $R$ reads. $p_k$ is the probability of
transcript $k$, $k=1,\ldots,N$. $\theta=\{p_k\}_k$ is the set of parameters.
$Z_i\in\{1,\ldots,N\}$ is the true origin of read $i$. It is a latent variable.
Let $\alpha_{iz}\triangleq\Pr\{d_i|Z_i=z\}$ be the probability of read $i$
being generated from transcript $z$. For a simple model, let $l_z$ be the
\emph{effective} length of transcript $z$. We can compute $\alpha_{iz}=1/l_z$
if the hit to $z$ is one of the equally best hits of read $i$; otherwise
$\alpha_{iz}=0$.
\[
\Pr\{d_i,Z_i=z|\theta\}=\Pr\{d_i|Z_i=z\}\Pr\{Z_i=z|\theta\}=\alpha_{iz}p_z
\]
\[
\beta_{iz}(\theta)\triangleq\Pr\{Z_i=z|d_i,\theta\}=\frac{\alpha_{iz}p_z}{\sum_{y}\alpha_{iy}p_y}
\]
$\beta_{iz}$ is the posterior probability that read $i$ comes from transcript
$z$. The likelihood function is
$$
\mathcal{L}(\theta)=\sum_{\bf z}\Pr\{{\bf d},{\bf Z}={\bf z}|\theta\}=\sum_{z_1=1}^N\cdots\sum_{z_R=1}^N\prod_{i=1}^R\Pr\{d_i,Z_i=z_i|\theta\}
=\sum_{z_1=1}^N\cdots\sum_{z_R=1}^N\prod_{i=1}^R\alpha_{iz_i}p_{z_i}
$$
There are $R^N$ terms. It is difficult to directly derive the MLE of $p_z$.
We now solve this problem with EM. As a preparation
\[
\log\Pr\{{\bf d},{\bf Z}={\bf z}|\theta\}=\log\Pr\{{\bf d}|{\bf Z}={\bf z}\}\Pr\{{\bf Z}={\bf z}|\theta\}=C+\sum_i\log p_{z_i}
\]
where $C$ is not a function of parameters $\theta$.

We now compute the $Q$ function:
\begin{eqnarray*}
Q(\theta|\theta^t)&=&\sum_{{\bf z}}\Pr\{{\bf Z}={\bf z}|{\bf d},\theta^t\}\cdot\log\Pr\{{\bf d},{\bf Z}={\bf z}|\theta\}\\
&=&C+\sum_{{\bf z}}\prod_j\Pr\{Z_j=z_j|d_j,\theta^t\}\sum_i\log p_{z_i}\\
&=&C+\sum_{i=1}^R\sum_{z_i=1}^N\Pr\{Z_i=z_i|d_i,\theta^t\}\log p_{z_i}\\
&=&C+\sum_{i=1}^R\sum_{z=1}^N\beta_{iz}(\theta^t)\log p_z
\end{eqnarray*}
Under constraint $\sum_zp_z=1$, requiring
$\partial_{p_y}(Q-\lambda\sum_zp_z)=0$ leads to
\[
p_y=\frac{1}{\lambda}\sum_i\beta_{iy}(\theta^t)
\]
and thus
\[
\lambda=\sum_i\sum_z\beta_{iz}(\theta^t)=R
\]
Then
\[
p_y^{(t+1)}=\frac{1}{R}\sum_i\beta_{iy}(\theta^t)=\frac{1}{R}\sum_i\frac{\alpha_{iz}p^{(t)}_z}{\sum_{y}\alpha_{iy}p^{(t)}_y}=\frac{1}{R}\sum_i\frac{p_z^{(t)}/l_z}{\sum_yp_y^{(t)}/l_y}
\]
\section{Markov Chain}

\subsection{Markov chain: the basic}
Let $\Sigma$ be an alphabet and $a\in\Sigma$ be a symbol. Markov property:
$$
\Pr\{X_i=a_i|X_{1,i-1}=a_1\cdots a_{i-1}\}=\Pr\{X_i=a_i|X_{i-1}=a_{i-1}\}
$$
A Markov chain is \emph{homogeneous} if
$$
\Pr\{X_i=a_i|X_{i-1}=a_{i-1}\}=\Pr\{X_j=a_j|X_{j-1}=a_{j-1}\}
$$
for any $i$ and $j$. Let $p_{ab}=p(b|a)\triangleq\Pr\{X_i=b|X_{i-1}=a\}$ be the
transition probability from $a$ to $b$. Then
$$
P(x)=\Pr\{X_{1,L}=a_1\cdots a_L\}=q(a_1)\prod_{i=2}^Lp(a_i|a_{i-1})
$$
where $q(a)$ is the initial probability. It is \emph{stationary} if
$$
q(a)=\sum_{b\in\Sigma} q(b)p(a|b)
$$

Suppose there are $n$ states. Define square transition matrix
$$
\mathbf{P}=\left(\begin{array}{cccc}
p(1|1) & p(2|1) & \cdots & p(n|1) \\
p(1|2) & p(2|2) & \cdots & p(n|2) \\
\vdots & \vdots & \ddots & \vdots \\
p(1|n) & p(2|n) & \cdots & p(n|n)
\end{array}\right)
$$
and column vector
$$
\mathbf{q}=\left(\begin{array}{c}
q(1) \\
q(2) \\
\vdots\\
q(n)
\end{array}\right)
$$
Then in the matrix form:
$$
\mathbf{q}^\intercal\cdot\mathbf{P}=\mathbf{q}^\intercal
$$

\subsection{$k$-order Markov chain}
$$
\Pr\{X_i=a_i|X_{1,i-1}=a_1\cdots a_{i-1}\}=\Pr\{X_i=a_i|X_{i-k,i-1}=a_{i-k}\cdots a_{i-1}\}
$$

\section{Hidden Markov Model (HMM)}
\subsection{HMM: the basic}
Let $\mathcal{S}$ be the state space. A homogeneous HMM is defined by:
$$
p(t|s)\triangleq\Pr\{Z_i=t|Z_{i-1}=s\}
$$
$$
e(a|s)\triangleq\Pr\{X_i=a|Z_i=s\}
$$

\subsubsection{The Viterbi algorithm}
$$
\gamma(i+1,t)=e(a_{i+1}|t)\max_s\gamma(i,s)p(t|s)
$$

\subsubsection{The forward algorithm}
\begin{eqnarray*}
\alpha(i,t)&\triangleq&\Pr\{Z_i=t,X_{1,i}=a_1\cdots a_i\}\\
&=&\sum_{s\in\mathcal{S}}\Pr\{Z_i=t,Z_{i-1}=s,X_{1,i}=a_1\cdots a_i\}\\
&=&\sum_s\Pr\{Z_i=t,X_i=a_i,Z_{i-1}=s,X_{1,{i-1}}=a_1\cdots a_{i-1}\}\\
&=&\sum_s\Pr\{Z_i=t,X_i=a_i|Z_{i-1}=s\}\Pr\{Z_{i-1}=s,X_{1,{i-1}}=a_1\cdots a_{i-1}\}\\
&=&\sum_s\Pr\{Z_i=t|Z_{i-1}=s\}\Pr\{X_i=a_i|Z_i=t\}\alpha(i-1,s)\\
&=&e(a_i|t)\sum_s\alpha(i-1,s)p(t|s)
\end{eqnarray*}
For convenience and consistency, we often assume at $i=0$, the HMM starts at
state $\epsilon$ at probability 1, i.e. $\Pr\{Z_0=\epsilon\}=1$. Then at $i=1$:
$$
\alpha(1,t)=\Pr\{Z_1=t,X_1=a_1\}=\Pr\{Z_1=t,X_1=a_1|Z_0=\epsilon\}=e(a_1|t)p(t|\epsilon)
$$
where $p(t|\epsilon)$ is arbitrary. In practice, we often choose
$p(t|\epsilon)=q(t)$.

The probability of a sequence
$$
P(x)=\Pr\{X_{1,L}=a_1\cdots a_L\}=\sum_s\Pr\{X_{1,L}=a_1\cdots a_L,Z_i=s\}=\sum_s\alpha(i,s)
$$

\subsubsection{The backward algorithm}
\begin{eqnarray*}
\beta(i,s)&\triangleq&\Pr\{X_{i+1,L}=a_{i+1}\cdots a_L|Z_i=s\}\\
&=&\sum_{t\in\mathcal{S}}\Pr\{X_{i+1,L}=a_{i+1}\cdots a_L,Z_{i+1}=t|Z_i=s\}\\
&=&\sum_t\Pr\{X_{i+1,L}=a_{i+1}\cdots a_L|Z_{i+1}=t\}\Pr\{Z_{i+1}=t|Z_i=s\}\\
&=&\sum_t\Pr\{X_{i+2,L}=a_{i+2}\cdots a_L|Z_{i+1}=t\}\Pr\{X_{i+1}=a_{i+1}|Z_{i+1}=t\}p(t|s)\\
&=&\sum_t\beta(i+1,t)e(a_{i+1}|t)p(t|s)
\end{eqnarray*}
Similarly, we assume all sequences end up with a special symbol
``{\tt\char36}'', i.e. $X_{L+1,L+1}=a_{L+1}={\tt\char36}$. Then
$$
\beta(L,s)=\Pr\{X_{L+1,L+1}={\tt\char36}|Z_L=s\}=1
$$

\subsubsection{Posterior decoding}
\begin{eqnarray*}
&& \Pr\{X_{1,L}=a_1\cdots a_L,Z_i=s\}\\
&=&\Pr\{X_{i+1,L}=a_{i+1}\cdots a_L,X_{1,i}=a_1\cdots a_i,Z_i=s\}\\
&=&\Pr\{X_{i+1,L}=a_{i+1}\cdots a_L|Z_i=s\}\Pr\{Z_i=s,X_{1,i}=a_1\cdots a_i\}\\
&=&\alpha(i,s)\beta(i,s)
\end{eqnarray*}
$$
\Pr\{Z_i=s|x\}=\frac{\alpha(i,s)\beta(i,s)}{P(x)}
$$

\end{document}
