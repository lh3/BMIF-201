\documentclass{article}[10pt]

\usepackage{amsthm}
\theoremstyle{definition}\newtheorem{question}{Question}
\renewcommand*{\proofname}{Solution}

\usepackage[margin=1in,footskip=0.25in]{geometry}

\usepackage{hyperref}

\renewcommand{\ttdefault}{cmtt}

\begin{document}

\begin{question}
Given a string $T$, a \emph{subsequence} of $T$ is composed of symbols in $T$
without changing their original order in $T$. For example, both ${\tt CGA}$ and
${\tt CCT}$ are subsequences of ${\tt CGCTA}$. Given two strings $T_1$ and
$T_2$, a \emph{common subsequence} is a string that is the subsequence of both
$T_1$ and $T_2$. For example, ${\tt CGA}$ is a common subsequence of ${\tt
CGCTA}$ and ${\tt CTGAT}$. The 2-string longest common subsequence problem
is to find the longest common subsequence (LCS) among all subsequences between
two strings. Questions:
\begin{enumerate}
\item Find the length of LCS between two strings with dynamic programming (DP).
Write down the DP equation. \emph{Hint: very similar to computing edit distance.}
\item Implement the algorithm and compute the length of LCS between 
``${\tt GCATAAATTATAACTCCAGCTATAGC}$'' and ``${\tt CAAACCCCAGGGACGGGTTAGCA}$''.
Try to use less memory.
\end{enumerate}
\end{question}

\begin{proof}
Given two strings $A=a_1a_2\ldots a_m$ and $B=b_1b_2\ldots b_n$, let $f(i,j)$
be the length of LCS between prefixes $a_1a_2\ldots a_i$ and $b_1b_2\ldots
b_j$. Then $f(m,n)$ gives the length of LCS between $A$ and $B$.

The computation of $f()$ starts with $f(i,0)=0$ and $f(0,j)=0$ for $0\le i\le m$
and $0\le j\le n$. The DP equation is
$$
f(i,j)=\max\{f(i-1,j-1)+\delta(a_i,b_j),f(i-1,j),f(i,j-1)\}
$$
where $\delta(a,b)$ equals 1 if $a=b$, or 0 otherwise.

A python implementation can be found at
\href{https://ideone.com/Om4JhD}{https://ideone.com/Om4JhD}. For the two given
strings, the length of LCS is 15.

An equivalent equation is
$$
f(i,j)=\left\{\begin{array}{ll}
0            & \mbox{if $i=0$ or $j=0$}\\
f(i-1,j-1)+1 & \mbox{if $i,j>0$ and $a_i=b_j$}\\ 
\max\{f(i-1,j),f(i,j-1)\} & \mbox{otherwise}
\end{array}\right.
$$
It is equivalent because $f(i-1,j-1)\le\max\{f(i-1,j),f(i,j-1)\}$.
\end{proof}



\begin{question}
Suppose there are $n$ distinct $k$-mers in two DNA sequences and $m$ of which are
shared between the two sequences. Questions:
\begin{enumerate}
\item Prove that the similarity between the two sequences can be estimated by
the following equation, assuming no $k$-mers occurring multiple times on one
sequence.
$$
\hat{p}=\left(\frac{2m}{n+m}\right)^{1/k}
$$
\item Implement the algorithm and find the sequence similarity between the following two strings for $k=7$:
\begin{quote}
\footnotesize\tt
CGATATAAGCAATATTATTATATTACGCCCAATAAACGATAGCTAATATGCCGCGGAGCGATCGAACCGCCCCGATAAGACCCC\\
CGATATAAGCAATATTAaTATATTAGCCCAATAAACGATAGCTAATATGggGCGGAGCGATCGAACCGCCCCGAaaTAAGACCCC
\end{quote}
\item Discuss potential problems with this approximation.
\end{enumerate}
\end{question}

\begin{proof}
If there are no $k$-mers occurring multiple times in one sequence, there are
$n+m$ $k$-mers in total in the two sequences and $2m$ of them are shared.
Let the similarity be $p$. The probability of seeing no differences on a
$k$-long sequence is $p^k$. The likelihood function is
$$
\mathcal{L}(p)=\left(p^k\right)^{2m}\left(1-p^k\right)^{n+m-2m}=p^{2km}\left(1-p^k\right)^{n-m}
$$
Therefore,
$$
\frac{\partial\log\mathcal{L}(p)}{\partial p}=\frac{2km}{p}-\frac{n-m}{1-p^k}\cdot kp^{k-1}=\frac{k\cdot\left[2m-(n+m)p^k\right]}{p\left(1-p^k\right)}
$$
The maximum likelihood estimate of $p$ is obtained at
$\partial\log\mathcal{L}/\partial p=0$. Solving the equation gives
$$
\hat{p}=\left(\frac{2m}{n+m}\right)^{1/k}
$$

An alternative derivation can use a Poisson approximation. In this case, the
probability of seeing no differences on a $k$-mer is $e^{-pk}$. The maximum
likelihood estimate is $-\frac{1}{k}\log\frac{2m}{m+n}$. We can find this
approximation in the Mash paper (Ondov et al, 2016). The first derivation is
preferred as it uses one less approximation.

A sorting-based python implementation is available at
\href{https://ideone.com/F02WCa}{https://ideone.com/F02WCa}. The 7-mer
similarity is 93.8\%. This algorithm can also implemented with a hash table.
The major problem with this approximation is that it assumes all $k$-mers are
independent. Due to actual $k$-mer dependencies, the estimate varies with $k$.
\end{proof}



\begin{question}
Suppose there are three transcripts of 200bp, 500bp and 300bp in length, respectively.
Five reads are mapped to these transcripts. Read A and B are mapped to transcript 1 and 2,
read C and D to transcript 2 only and read E is mapped to transcript 3 only.
Use EM to estimate the transcript fraction of the three transcripts.
\end{question}

\begin{proof}
25\%, 50\% and 25\%. Source code: \href{https://ideone.com/F02WCa}{https://ideone.com/F02WCa}.
\end{proof}



\begin{question}
Implement a simplified CNV caller using HMM. Suppose after read mapping, we
symbolize a series of loci into a sequence like ``${\tt
11111110102222211110111}$'' where symbol `${\tt 1}$' for normal coverage,
`${\tt 0}$' for lower than expected coverage and `${\tt 2}$' for higher
coverage. We design a 2-state HMM to call duplications on the symbol sequence.
The HMM has two hidden states: `${\tt N}$' for normal ploidy and `${\tt D}$' for
duplications. We parameterize the HMM as follows, where $P$ gives the transition
probabilities between hidden states and $E$ gives the emission probabilities:
$$
{\bf P}=\left(\begin{array}{ll}
p({\tt N}|{\tt N}) & p({\tt D}|{\tt N}) \\
p({\tt N}|{\tt D}) & p({\tt D}|{\tt D})
\end{array}\right)
=\left(\begin{array}{ll}
0.9 & 0.1 \\
0.2 & 0.8
\end{array}\right)
$$
\end{question}
$$
{\bf E}=\left(\begin{array}{lll}
e({\tt 0}|{\tt N}) & e({\tt 1}|{\tt N}) & e({\tt 2}|{\tt N})\\
e({\tt 0}|{\tt D}) & e({\tt 1}|{\tt D}) & e({\tt 2}|{\tt D})
\end{array}\right)
=\left(\begin{array}{lll}
0.15 & 0.70 & 0.15\\
0.01 & 0.19 & 0.80
\end{array}\right)
$$
Questions:
\begin{enumerate}
\item Derive the probability of the hidden state if we run the HMM for infinite
time (i.e. the stationary distribution).
\item Write a program to infer the hidden states of symbol sequence with posterior decoding. 
\begin{quote}
\footnotesize\tt 1211100122111011112222112022010200111001211202220112112111220222112221211121002122122222222212222112\\
\end{quote}

Here is a sample output for a different input
\begin{quote}
\footnotesize\tt 1012111121101211001111121120012222222222212221220111211111111002212222221112212122111111120111222122
\footnotesize\tt NNNNNNNNNNNNNNDDNNNNNNNNNNNNNNDDDDDDDDDDDDDDDDDDNNNNNNNNNNNNNNNDDDDDDDDNNNDDDDDNDDNNNNNNNNNNNNDDDDDD
\end{quote}
\end{enumerate}

\begin{proof}
The stationary distribution ${\bf q}=(q({\tt N}),q({\tt D}))$ follows
$$
{\bf q}\cdot{\bf P}={\bf q}
$$
on the condition that $q({\tt N})+q({\tt D})=1$. Solving this equation gives
$$
q({\tt N})=\frac{p({\tt N}|{\tt D})}{p({\tt N}|{\tt D}) + p({\tt D}|{\tt N})}
$$
In this example, $q({\tt N})=2/3$.

For the given string, the inferred hidden states are:
\begin{quote}
\footnotesize\tt 1211100122111011112222112022010200111001211202220112112111220222112221211121002122122222222212222112\\
\footnotesize\tt NNNNNNNNDDNNNNNNNNDDDDNNNNDNNNNNNNNNNNNNNNNNNDDDNNNNNNNNNNDDNDDDDDDDDDDNNNNNNNNNDDDDDDDDDDDDDDDDDNNN
\end{quote}
A python implementation is available at \href{https://ideone.com/nnh6Q2}{https://ideone.com/nnh6Q2}.

\begin{flushleft}
PS: This is simulated data. The truth actually is
\end{flushleft}
\begin{quote}
\footnotesize\tt 1211100122111011112222112022010200111001211202220112112111220222112221211121002122122222222212222112\\
\footnotesize\tt NNNNNNNNNDDNNNNNNNNDDDNNNNNNNNNNNNNNNNNNNNNNNDDDNNNNNNNNNDDDNDDDDDDDDDDNNNNNNNDDDDDDDDDDDDDDDDDDDDDD
\end{quote}

The simulator and the posterior decoder can be found at
\href{https://ideone.com/a9a2h0}{https://ideone.com/a9a2h0}. Students do not
need to provide a simulator.

\end{proof}

\end{document}
